\chapter{Introduzione}
\section{Scopo del documento}
Lo scopo di questo documento é presentare una descrizione dettagliata del 
prodotto software DietiEstates25. Verranno illustrati: 
+ i requisiti che il prodotto software dovrá soddisfare e i metodi con cui questi sono stati raccolti.
+ l'architettura del sistema.
+ la metodologia ed i criteri di testing.
Questo documento é rivolto al committente ed in generale agli stakeholders. 

\section{Scopo del progetto}
DietiEstates25 é un software che:
\begin{list}{$\cdot$}{}
    \item permette alle agenzie immobiliari di gestire in modo facile,
    completo ed efficace il proprio business.
    \item permette alle persone di trovare rapidamente l'immobile 
    che desiderano.
\end{list}

\section{Glossario}
\begin{center}
\begin{tabular}{| m{8em} | m{8cm}|}
    \hline
    Manager & Un componente di amministrazione dell'agenzia immobiliare 
    che si occupa di gestire gli agenti dell'agenzia. \\
    \hline
    Administrator & L'unico manager con privilegi. E' gerarchicamente 
    elevato rispetto agli altri manager.
    \\
    \hline
    Agent & Un agente immobiliare appartenente ad un'agenzia immobiliare 
    che si occupa di gestione degli annunci e delle relazioni con i clienti (customers). \\
    \hline
    Property & Un bene immobile, ovvero un bene stabilmente fisso che non è 
    possibile spostare, che può essere sia un fabbricato che un terreno.\\
    \hline
    Listing & Un annuncio pubblicato da un agente che descrive la messa in 
    vendita/affitto di un immobile. \\
    \hline
    Customer & Una persona che desidera acquistare/affittare un immobile.\\
    \hline
    Tour & Visita dell'immobile concordata tra cliente ed agente al fine del 
    suo eventuale acquisto/affitto.\\
    \hline
    Personal data & I dati che un utente ha inserito in fase di registrazione: 
    e-mail, telefono, nome, cognome \dots \\
    \hline

\end{tabular}
\end{center}