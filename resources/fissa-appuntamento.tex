% Per tabelle che occupano più di una pagina:
%  longtblr al posto di tblr
\begin{longtblr}[
    caption = {Diagramma di Cockburn del caso d'uso \textit{Fissa Appuntamento}.}
]{
    column{1} = {3cm},
    column{2} = {0.8cm},
    column{3-5} = {3cm},
	vlines = {}, %per le cornici verticali
	hlines = {}, %per le cornici orizzontali
    % Merge delle righe/colonne:
    cell{1-7}{2} = {c = 4}{10cm, halign = l},
    cell{8}{1} = {r = 9}{valign = t},
    cell{17}{1} = {r = 3}{valign = t},
    cell{20}{1} = {r = 3}{valign = t},
    cell{23}{1} = {r = 3}{valign = t}
}
USE CASE & Fissa appuntamento & & & \\
Goal in Context & Il Cliente vuole fissare un appuntamento con un Agente per la visione dell’immobile di interesse. & & & \\
Preconditions & Il Cliente deve essersi autenticato. & & & \\
Success End Condition & Il sistema tiene traccia della richiesta di appuntamento da parte del Cliente e la invia all’Agente. & & & \\
Fail End Condition & Il sistema non tiene traccia dell’attività non completata. & & & \\
Primary Actor & Cliente & & & \\
Trigger & Il Cliente preme il pulsante “Richiedi appuntamento” di \ref{fig:M-FA1}. & & & \\
Main Scenario   & Step & Cliente & Agente & System \\
 & 1 & Trigger action. & & \\
 & 2 & & & Mostra \ref{fig:M-FA2}. \\
 & 3 & Seleziona le sue preferenze. Clicca su Prosegui. & & \\
 & 4 & & & Mostra \ref{fig:M-FA3}. \\
 & 5 & Clicca su “Invia richiesta”. & & \\
 & 6 & & & Notifica al Cliente l'invio della richiesta. \\
 & 7 & & Accetta la richiesta. & \\
 & 8 & & & Notifica al Cliente la conferma dell’appuntamento e termina lo use case. \\
 
\pagebreak
Extension A & Step & Cliente & Agente & System \\
 & 3.a & Clicca su “Annulla”. & & \\
 & 4.a & & & Mostra \ref{fig:M-FA1} e termina use case.  \\
Extension B & Step & Cliente & Agente & System \\
 & 5.b & Clicca su “Annulla”. & & \\
 & 6.b & & & Torna allo step 2 del main scenario. \\
Extension C & Step & Cliente & Agente & System \\
 & 7.c & & Rifiuta la richiesta. & \\
 & 8.c & & & Invia al Cliente la notifica di rifiuto dell’appuntamento e termina lo use case. \\
\end{longtblr}