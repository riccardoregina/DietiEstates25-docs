% Per tabelle che occupano più di una pagina:
%  longtblr al posto di tblr
\begin{longtblr}[
    caption = {Diagramma di Cockburn del caso d'uso \textit{Ricerca Immobili}.}
]{
    width = 10cm, %per fissare una larghezza
    column{1} = {3cm},
    column{2} = {0.8cm},
    column{3-4} = {3cm},
	vlines = {}, %per le cornici verticali
	hlines = {}, %per le cornici orizzontali
    % Merge delle righe/colonne:
    cell{1-7}{2} = {c = 3}{10cm, halign = l},
    cell{8}{1} = {r = 7}{valign = t},
    cell{15}{1} = {r = 3}{valign = t},
    cell{18}{1} = {r = 5}{valign = t},
    cell{23}{1} = {r = 2}{valign = t},
    cell{25}{1} = {r = 3}{valign = t},
    cell{28}{1} = {r = 3}{valign = t},
    cell{31}{1} = {r = 3}{valign = t}
}
USE CASE & Ricerca Immobili & & \\
Goal in Context & Il Cliente vuole trovare annunci di immobili di suo interesse. & & \\
Preconditions & Nessuna precondizione. & & \\
Success End Condition & Il sistema tiene traccia della ricerca effettuata, salvandola tra le “Ricerche recenti”. & & \\
Fail End Condition & Il sistema tiene traccia della ricerca effettuata, salvandola tra le “Ricerche recenti”. & & \\
Primary Actor & Cliente & & \\
Trigger & Il Cliente clicca sulla barra di ricerca di \ref{fig:M-RI1}. & & \\
Main Scenario & Step & Cliente & Sistema   \\
 & 1 & Trigger & \\
 & 2 & & Mostra \ref{fig:M-RI2}. \\
 & 3 & Inserisce testo nella barra di ricerca.
 Seleziona il tipo di annuncio.
 Clicca su “Cerca”. & \\
 & 4 & & Mostra \ref{fig:M-RI3} \\
 & 5 & Clicca su un risultato. & \\
 & 6 & & Mostra \ref{fig:M-RI4} Termina use case. \\
Extension A: 
click su una ricerca recente & Step & Cliente & Sistema \\
 & 3.a & Clicca su una ricerca recente. & \\
 & 4.a & & Torna allo step 4 del main scenario. \\
Extension B: 
ricerca tramite punto sulla mappa. & Step & Cliente & Sistema \\
 & 3.b & Clicca su “Ricerca tramite punto sulla mappa”. & \\
 & 4.b & & Mostra la schermata di ricerca tramite punto sulla mappa. \\
 & 5.b & Seleziona un punto sulla mappa, specificando un raggio di ricerca e clicca conferma. & \\
 & 6.b & & Torna allo step 4 del main scenario. \\

\pagebreak
Extension C:
la ricerca non ha prodotto risultati. & Step & Cliente & Sistema \\
 & 4.c & & Mostra \ref{fig:M-RI3bis}. Torna allo step 2 del main scenario. \\
Extension D: nessun testo inserito.
 & Step & Cliente & Sistema \\
 & 3.d & non inserisce testo nella barra di ricerca e clicca su “Cerca”. & \\
 & 4.d & & Mostra messaggio “Inserisci una zona o un indirizzo”. Torna allo step 2 del main scenario. \\
Extension E: applica filtri. & Step & Cliente & Sistema \\
 & 5.e & Modifica dei filtri di ricerca. & \\
 & 6.e & & Torna allo step 4 del main scenario. \\
Extension F: annulla ultima operazione & Step & Cliente & Sistema \\
 & 5.b.f & Clicca sul pulsante annulla & \\
 & 6.b.f & & Torna allo step 2 del main scenario. \\
\end{longtblr}